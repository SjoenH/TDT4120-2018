\documentclass{article}
    % \usepackage{biber}
    \usepackage{listings}
    \usepackage{qtree}
        
\author{Henry S. Sjoen}
\title{TDT4120 \\ Fall 2018}

\begin{document}
%.root left right 
\Tree[.Top 
    [ .Left   
        [.Left   Left  Right ] 
        [.Right  Left  Right ]
    ]  
    [ .Right   
        [.Left   Left  Right ] 
        [.Right  Left  Right ]
    ]
]

\maketitle
\tableofcontents
\section{Algorithm Design}


\subsection{Divide and Conquer}
Divide and Conquer is an Algorithm design paradigm based on multi-branch recursion. A divide and conquer algorithm works by recursively breaking down a problem into two or more sub-problems of the same or related type, until these become simple enough to be solved directly. The solutions to the sub-problems are then combined to give a solution to the original problem.
% \cite{wikiDivide}

\section{Loose Recurrences}
\subsection{The Master-Theorem}
% https://en.wikipedia.org/wiki/Master_theorem_(analysis_of_algorithms)

Generic form 
$T(n)=a T(\frac{n}{b})+f(n)$

\subsection{Recursion Trees}
$T(n)=4T(\frac{n}{2})+n^2$

%.root left right 
\Tree[.$n^2$ 
    [ .$(\frac{n}{2})^2$   
        [.$(\frac{n}{4})^2$  $(\frac{n}{8})^2$  $(\frac{n}{8})^2$ ] 
        [.$(\frac{n}{4})^2$  $(\frac{n}{8})^2$  $(\frac{n}{8})^2$ ]
    ]
    [ .$(\frac{n}{2})^2$   
        [.$(\frac{n}{4})^2$  $(\frac{n}{8})^2$  $(\frac{n}{8})^2$ ] 
        [.$(\frac{n}{4})^2$  $(\frac{n}{8})^2$  $(\frac{n}{8})^2$ ]
        ]
    ]

% $(n\frac{n}{8}^2$
\begin{lstlisting}
This is supposed to be a tree...

n^2
-(n/2)^2
--(n/4)^2
--(n/4)^2
--(n/4)^2
--(n/4)^2
-(n/2)^2
--(n/4)^2
--(n/4)^2
--(n/4)^2
--(n/4)^2
-(n/2)^2
--(n/4)^2
--(n/4)^2
--(n/4)^2
--(n/4)^2
-(n/2)^2
--(n/4)^2
--(n/4)^2
--(n/4)^2
--(n/4)^2
\end{lstlisting}

\subsection{Variable-switching}
$T(n) = 2T(\sqrt{n})+\log n$


\section{Sorting Algorithms}
\subsection{Mergesort}
$\theta(n \log n)$

\subsection{Quicksort}
Howto:

\begin{lstlisting}[caption=Julia example]
    function traverse_recursive_max(node, start_value)
        highest_value = start_value
        if (node.value > highest_value)
            highest_value = node.value
        end
        if node.next == nothing
            return highest_value
        end
        traverse_recursive_max(node.next,highest_value)
    end
    
    traversemax(node) = traverse_recursive_max(node,node.value)
\end{lstlisting}


\section{Code example overview}
\lstlistoflistings
\end{document}